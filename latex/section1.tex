
\documentclass[a4paper]{report}

\begin{document}

\section*{Section 1 (60\%)}

\subsubsection*{What is the generator polynomial for this code? Use MATLAB’s ’bchgenpoly’ function to verify your answer}

The generator polynomial for a n = 31, k = 16, and t = 3 BCH code is $g(x) = x^{15} + x^{11} + 
x^{10} + x^{9} + x^{8} + x^{7} + x^{5} + x^{3} + x^{2} + x^{1} + 1$. 
This was verified with \textit{[gen, t] = bchgenpoly(31,16)} \\

\subsubsection*{What is the minimum distance of this code?}

For a BCH code, $d_{min} \geq 2*t + 1 = 7$ \\

\subsubsection*{Construct the reduced syndrome lookup table for this code. You need
to write a program to do this since it is difficult to do it by hand. You
do not need to include the whole array in your report due to its large
size. Instead, just show the sub-array consisting of the first 5 rows in your
report, and include a separate text file (*.txt) enumerating all the data in
your electronic submission.}

Generate table by: \\
\textit{[h g k] = cyclgen(31, gen)}\\
\textit{t = syndtable(h)}\\

returns cosets is a "reduced syndrome lookup table"??? - Check \\
Think this is wrong ... shouldn't there be 64 rows? ($2^6$)?? \\

The first 5 rows are: \\
\begin{tabular}{{| c}*{16} |}
0 & 0 & 0 & 0 & 0 & 0 & 0 & 0 & 0 & 0 & 0 & 0 & 0 & 0 & 0 & 0 & 0 & 0 & 0 & 0 & 0 & 0 & 0 & 0 & 0 & 0 & 0 & 0 & 0 & 0 & 0 \\
0 & 0 & 0 & 0 & 0 & 0 & 0 & 0 & 0 & 0 & 0 & 0 & 0 & 0 & 1 & 0 & 0 & 0 & 0 & 0 & 0 & 0 & 0 & 0 & 0 & 0 & 0 & 0 & 0 & 0 & 0 \\
0 & 0 & 0 & 0 & 0 & 0 & 0 & 0 & 0 & 0 & 0 & 0 & 0 & 1 & 0 & 0 & 0 & 0 & 0 & 0 & 0 & 0 & 0 & 0 & 0 & 0 & 0 & 0 & 0 & 0 & 0 \\
0 & 0 & 0 & 0 & 0 & 0 & 0 & 0 & 0 & 0 & 0 & 0 & 0 & 1 & 1 & 0 & 0 & 0 & 0 & 0 & 0 & 0 & 0 & 0 & 0 & 0 & 0 & 0 & 0 & 0 & 0 \\
0 & 0 & 0 & 0 & 0 & 0 & 0 & 0 & 0 & 0 & 0 & 0 & 1 & 0 & 0 & 0 & 0 & 0 & 0 & 0 & 0 & 0 & 0 & 0 & 0 & 0 & 0 & 0 & 0 & 0 & 0 \\
\end{tabular}
Answer here - the complete file is in \textbf{Appendix A} - NB: save later\\

\subsubsection*{Based on the standard array you obtained in task 3, find out the weight distribution of the coset leaders.}

Answer here \\

\subsubsection*{Encoding: Design and implement an encoder using the generator polynomial for this BCH code.}

Answer here \\

\subsubsection*{Decoding: a) Use MATLAB defined functions (eg. the ”decode” function) to decode the BCH code.}

Answer here \\

\subsubsection*{Decoding: b) Design and implement a decoder using the syndrome decoding table in Task 3 for the BCH code.}

Answer here \\

\subsubsection*{Decoding: c) Design and implement a decoder using the Berlekamp’s iterative procedure.}

Answer here \\

\subsubsection*{Simulations and sound analysis: Simulate the implemented BCH encoder and decoder (using ANY method in Part 6) using the attached wave file (austinpowers.wav) in a BSC channel for different transition probabilities. Discuss the impact of changing different transition probability values.}

Answer here \\

\subsubsection*{Simulations: a) Simulation over BSC Simulate the BCH code in a BSC channel (you are allowed to use MATLAB defined functions) and plot the BER versus transition probability for coded and uncoded systems on the same graph. Plot BER versus $\frac{E_b}{N_0}$ for coded and uncoded systems on the same graph by assuming that the SNR ($ = \frac{E_b}{N_0}$ ) is related to the transition probability for the coded system via $X_{dB,coded} = 10 \log_{10}(\frac{[Q^{−1} (p)]^2}{R}) $ and that for uncoded system is $X_{dB,uncoded} = 20\log_{10}(Q^{−1} (p)) $, where $Q(x) = \frac{1}{\sqrt{2 \pi}} \int\limits_{x}\limits^{+\infty} e^{ -\frac{z^2}{2}} dz $ and $0 < p < 0.5$.}

Answer here \\

\subsubsection*{Simultations: b) Simulation over AWGN channel with BPSK modulation ($\{−1, 1\}$) Simulate the BCH code in an AWGN channel (you are allowed to use MATLAB defined functions) and plot the BER versus the signal to noise ratio (SNR) for a BPSK coded and uncoded
systems on the same graph by using a hard-decision demodulator and binary decoder.}

Answer here \\

\subsubsection*{Simulations: c) Draw a table detailing the coding gain for BER$= [10^{−2}, 10^{−3}, 10^{−4} , 10^{−5} , 10^{−6} ]$ by reading the differences between the BER curves for BPSK coded and uncoded systems which have been obtained from simulations in part (a) and (b).}

\begin{tabular}{| c | c | c |}
\hline
BER & Coding Gain (Hard decision) & Coding Gain (BSC) \\
\hline
$10^{-2}$ & & \\
\hline
$10^{-3}$ & & \\
\hline
$10^{-4}$ & & \\
\hline
$10^{-5}$ & & \\
\hline
$10^{-6}$ & & \\
\hline
\end{tabular}

\subsubsection*{Simulations: d) Find the asymptotic coding gain when $\frac{Eb}{N0}$ is very large from the formula which is given in the lecture notes and compare with simulation results.}

Answer here \\

\section*{Section 2 (50\%)}

\subsubsection*{You are an engineer whose job is to design and analyze the performance of error
control codes for different clients. Client 2 requires a code for satellite transmission of digital TV. The satellite is power limited and very high reliability is required (as close to Shannon-capacity as possible). Low decoding complexity is desired, but is not essential. To meet these requirements an LDPC code is used}

Answer here \\

\subsubsection*{Justify LDPC design}

Answer here \\

\subsubsection*{Simulations and Sound analysis: Simulate the chosen code using the attached wave file (austinpowers.wav) in a BSC channel for different transition probabilities (you are allowed to use MATLAB defined functions.}

Answer here \\

\subsubsection*{Simulations and Sound analysis: Discuss the difference in sound quality compared to the BCH code in Section I, for different transition probabilities.}

Answer here \\

\subsubsection*{Simulations: a) Simulation over BSC Simulate the BCH code in a BSC channel (you are allowed to use MATLAB defined functions) and plot the BER versus transition probability for coded and uncoded systems on the same graph. Plot BER versus $\frac{E_b}{N_0}$ for coded and uncoded systems on the same graph by assuming that the SNR ($ = \frac{E_b}{N_0}$ ) is related to the transition probability for the coded system via $X_{dB,coded} = 10 \log_{10}(\frac{[Q^{−1} (p)]^2}{R}) $ and that for uncoded system is $X_{dB,uncoded} = 20\log_{10}(Q^{−1} (p)) $, where $Q(x) = \frac{1}{\sqrt{2 \pi}} \int\limits_{x}\limits^{+\infty} e^{ -\frac{z^2}{2}} dz $ and $0 < p < 0.5$.}

Answer here \\

\subsubsection*{Simultations: b) Simulation over AWGN channel with BPSK modulation ($\{−1, 1\}$) Simulate the BCH code in an AWGN channel (you are allowed to use MATLAB defined functions) and plot the BER versus the signal to noise ratio (SNR) for a BPSK coded and uncoded
systems on the same graph by using a hard-decision demodulator and binary decoder.}

Answer here \\

\subsubsection*{Simulations: c) Draw a table detailing the coding gain for BER$= [10^{−2}, 10^{−3}, 10^{−4} , 10^{−5} , 10^{−6} ]$ by reading the differences between the BER curves for BPSK coded and uncoded systems which have been obtained from simulations in part (a) and (b).}

\begin{tabular}{| c | c | c |}
\hline
BER & Coding Gain (Hard decision) & Coding Gain (BSC) \\
\hline
$10^{-2}$ & & \\
\hline
$10^{-3}$ & & \\
\hline
$10^{-4}$ & & \\
\hline
$10^{-5}$ & & \\
\hline
$10^{-6}$ & & \\
\hline
\end{tabular}

\subsubsection*{Simulations: d) Find the asymptotic coding gain when $\frac{Eb}{N0}$ is very large from the formula which is given in the lecture notes and compare with simulation results.}

Answer here \\

\end{document}